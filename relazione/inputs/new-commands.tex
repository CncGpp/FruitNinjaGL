% make writing commands easier
\usepackage{xparse}

% colored text
\usepackage{color}

% include eps, pdf graphics
\usepackage{graphicx}


% properly handle spaces after defines
\usepackage{xspace}



% ============================================================================ %
%                               NUOVI COMANDI                                  %
% ============================================================================ %
%\newcommand\MathFont[0]{}
\definecolor{probcolor}{RGB}{50, 50, 50}


% Helper di probabilità
\def\Phelper#1|#2\relax{\ifx\relax#2\relax\def\Pchoice{\Pone}\else\def\Pchoice{\Ptwo}\fi}
\def\Pone(#1){\left( #1 \right)}
\def\Ptwo(#1|#2){\left( #1 \,\middle|\, #2 \right)}
\def\pr{{\mathbf{pr}}}
\def\Pr{{\mathbf{Pr}}}

% Densità di probabilità
\makeatletter
\def\p{\@ifnextchar({\@p}{\@ifnextchar[{\@pn}{\pr}}}
\def\@pn[#1](#2){\pr{{\hskip -0.02em}#1}{\hskip -0.22em}\Phelper#2|\relax\Pchoice(#2)}
\def\@p(#1){\pr{\hskip -0.20em}\Phelper#1|\relax\Pchoice(#1)}
\makeatother

% Massa di probabilità
\makeatletter
\def\P{\@ifnextchar({\@P}{\@ifnextchar[{\@Pn}{\Pr}}}
\def\@Pn[#1](#2){\Pr{{\hskip -0.02em}#1}{\hskip -0.22em}\Phelper#2|\relax\Pchoice(#2)}
\def\@P(#1){\Pr{\hskip -0.20em}\Phelper#1|\relax\Pchoice(#1)}
\makeatother




% ============================================================================ %
% Funzioni e macro matematiche finalizzate alla notazione                      %
% ============================================================================ %
\newcommand{\mathsc}[1]{\text{\textsc{#1}}}                                    % maiuscoletto in mathmode

% __ NUMERI e ALGEBRA __
\newcommand{\s}[1]{\mathnormal{#1}}                                            % Stile per gli scalari
\renewcommand{\v}[1]{\boldsymbol{#1}}                                          % Stile per i vettori
\newcommand{\m}[1]{\v{#1}}                                                     % Stile per le matrici

\renewcommand{\r}[1]{\mathrm{#1}}
\newcommand{\sr}[1]{\r{#1}}                                                    % Stile per gli scalari
\newcommand{\vr}[1]{\boldsymbol{\r{#1}}}                                       % Stile per i vettori
\newcommand{\mr}[1]{\v{\r{#1}}}                                                % Stile per le matrici

\def\T{{\mathrm{T}}}													       % Trasposizione
\newcommand{\me}{\mathrm{e}}                                                   % numero di nepero

% __ PROBABILITA __
\newcommand*{\ind}{\perp}
\newcommand*{\cind}{\perp\!\!\!\perp}

\newcommand{\antecedents}[1]{#1^\set{A}}									   % Insieme dei nodi parent
\newcommand{\parents}[1]{#1^{\set{P}}}				                           % Insieme dei nodi parent diretti
\newcommand{\descendant}[1]{#1^\set{D}}									       % Insieme dei nodi children
\newcommand{\children}[1]{#1^{\set{C}}}			                               % Insieme dei nodi children diretti
\newcommand{\nonDescendant}[1]{#1^\set{N}}									   % Insieme dei nodi non discendenti
\newcommand*{\localIndipendences}{\mathbb{I}_\ell{\hskip -0.20em}(\mathcal{G})}% Indieme delle dipendenze locali di G


% __ INSIEMI e GRAFI __
\newcommand{\set}[1]{\mathbb{#1}}                                              % Stile insiemi
\newcommand{\Reals}{\set{R}}                                                   % Insieme dei numeri reali
\newcommand{\Integers}{\set{Z}}                                                % Insieme dei numeri interi
\newcommand{\Naturals}{\set{N}}                                                % Insieme dei numeri naturali



% __ OPERATORI MATEMATICI __
\newcommand{\dist}[1]{{\operatorname{dist}\!\left(#1\right)}}                  % Operatore di distanza
% Valore atteso
\makeatletter
\def\EX{\@ifnextchar\bgroup{\@EX}{\@ifnextchar[{\@@EX}{\mathbb{E}}}}
\def\@@EX[#1]#2{\mathbb{E}_{#1}\!\left[#2\right]}
\def\@EX#1{\mathbb{E}\!\left[{#1}\right]}
\makeatother

% Valore atteso.	usage: \expectation[y]{x}
\DeclareDocumentCommand \expectation { o m } {%
	\ensuremath{\mathbb{E}%
		\IfValueTF {#1} {%
			_{#1} \left[ #2 \right]%
		}{%
			\left[ #2 \right]%
		}%
	}\xspace%
}

\newcommand{\var}[1]{{\color{probcolor}\text{{\MathFont var}}\!\left(#1\right)}}           		% Varianza 
\newcommand{\cov}[1]{{\color{probcolor}\mathbf{\MathFont cov}\!\left[#1\right]}}           		% Covarianza   
\newcommand{\cor}[1]{{\color{probcolor}\mathbf{\MathFont cor}\!\left(#1\right)}}           		% Correlazione   
\newcommand{\val}[1]{{\color{probcolor}\text{\MathFont val}\!\left(#1\right)}} % Valori di una variabile aleatoria
\newcommand{\card}[1]{{\left|#1\right|}}                                       % Cardinalità 
\newcommand{\tr}[1]{{\text{\color{probcolor}{\MathFont tr}}\!\left(#1\right)}}       % Traccia
% Norma
\makeatletter
\def\norm{\@ifnextchar\bgroup{\@norm}{\@ifnextchar[{\@@norm}{\mathbb{E}}}}
\def\@@norm[#1]#2{\left\Vert{#2}\right\Vert_{#1}}
\def\@norm#1{\left\Vert{#1}\right\Vert}
\makeatother                                       


% __ COMPUTER SCIENCE __
\newcommand{\bigO}[1]{\mathcal{O}(#1)}                                         %Complessità asintotica  


% __ ML __
\newcommand{\dataset}[1]{\set{#1}}                                             %Per indicare un dataset

% __ ALTRO__
\newcommand*{\assumptionLabel}{\textbf{A\arabic*}}                             %Label da utilizzare


% ============================================================================ %
% Funzioni e macro matematiche di utiliy                                       %
% ============================================================================ %
\newcommand{\efrac}[2]{{\textsyle\frac{#1}{#2}}}              % exponet-fraction per avere il textsyle nelle frac
\newcommand*\rfrac[2]{{}^{#1}\!/_{#2}}                        % frazione compatta (invece di usare nicefrac)
\newcommand{\floor}[1]{\left\lfloor #1 \right\rfloor}         % Parentesi del floor
\newcommand{\ceil}[1]{\left\lceil #1 \right\rceil}            % Parentesi del ceil
\DeclareMathOperator*{\argmax}{arg\,max}
\DeclareMathOperator*{\argmin}{arg\,min}





% ============================================================================ %
% Macro per la velocizzazione dell'immissione                                  %
% ============================================================================ %
\newcommand{\ds}[0]{\displaystyle}                                             % DisplayStyle
\newcommand{\ts}[0]{\textstyle}                                                % TextStyle

\newcommand{\de}[0]{\mathrm{d}}                                                % d integrale

% Apici e pedici
\def\ij{{i,j}}
\def\ji{{j,i}}

\newcommand{\collectionof}[2]{{ {#1}_1, {#1}_2, \ldots, {#1}_{#2} }}           % collezione di oggetti
\newcommand{\setof}[2]{{\{ \collectionof{#1}{#2} \}}}                          % insieme di oggetti
\newcommand{\bagof}[2]{{ [\![ \collectionof{#1}{#2} ]\!] }}                    % multiinsieme di oggetti


% Abbreviazione di testi
\newcommand{\pmf}[0]{\textsc{p.m.f.}\phantom{ }}
\newcommand{\pdf}[0]{\textsc{p.d.f.}\phantom{ }}
\newcommand{\cdf}[0]{\textsc{c.d.f.}\phantom{ }}
\newcommand{\iid}[0]{\textsc{i.i.d.}\phantom{ }}


% __ FUNZIONI, DISTRIBUZIONI E DENSITÀ DI PROBABILITA __
% Simbolo distribuzione gauss. di parametri #1, #2
\def\normal(#1,#2){\mathcal{N}{\hskip -0.22em}\left(#1, #2\right)}        
% Simbolo distribuzione gauss. di variabile #1 parametri #2, #3
\def\normal(#1;#2,#3){\mathcal{N}{\hskip -0.22em}\left(#1;{\hskip 0.1em}#2,#3\right)}
% PDF gaussiana \normalpdf[<dimansionality>](<mu>,<sigma>)
\makeatletter
\def\normalpdf{\@ifnextchar({\@normalpdfU}{\@ifnextchar[{\@normalpdfM}{\Pr}}}
\def\@normalpdfM[#1](#2, #3){%
{\frac {\exp\!\left(-{\frac {1}{2}}({\v{x} }-{#2})^{\T }{#3}^{-1}({\v{x} }-{#2})\right)}{\sqrt {(2\pi )^{#1}|{#3}|}}}
}
\def\@normalpdfU(#1, #2){%
{\frac  {1}{#2 {\sqrt  {2\pi }}}}\;\me^{{-{\efrac  {\left(x-#1 \right)^{2}}{2#2 ^{2}}}}}
}%
\makeatother                                                                             


% Abbreviazioni variabili frequentemente usate
\newcommand*{\vx}{\v{x}}
\newcommand*{\vy}{\v{y}}
\newcommand*{\vz}{\v{z}}
\newcommand*{\vrx}{\vr{x}}
\newcommand*{\vry}{\vr{y}}
\newcommand*{\vrz}{\vr{z}}


% ============================================================================ %
% Modifiche grafiche                                                           %
% ============================================================================ %
\let\oldemptyset\emptyset													   % Memorizzo il vecchio emptyset
\let\emptyset\varnothing													   % Cambio il simbolo


% New definition of square root, closes over content
% Renames \sqrt as \oldsqrt
\let\oldsqrt\sqrt                                                              % Memorizzo il vecchio sqrt
% Defines the new \sqrt in terms of the old one                                % Cambio l'operatore
\def\sqrt{\mathpalette\DHLhksqrt}
\def\DHLhksqrt#1#2{%
	\setbox0=\hbox{$#1\oldsqrt{#2}$}\dimen0=\ht0
	\advance\dimen0-0.2\ht0
	\setbox2=\hbox{\vrule height\ht0 depth -\dimen0}%
	{\box0\lower0.4pt\box2}}



% Per ref boxati
\newcommand\refbox[2][]{%
	\tikz[overlay]%
	\node[align=center,
	      fill=blue!20,%
	      draw=none,
	      fill=baseBackground!98!black,
	      text=accentColor!95!black,
	      inner sep=2pt, %
	      anchor=text, %
	      rectangle, %
	      xshift=0.25em,
	      minimum height=0.45cm, 
	      rounded corners=1pt, #1] {{\hskip 0.05em}\textbf{#2}{\hskip 0.05em}};%
	\phantom{{\hskip 0.30em}\textbf{#2}{\hskip 0.30em}}%
}%

\newcommand\rref[1]{%
	\hyperref[#1]{%
		\refbox{%
			\ref*{#1}%
		}%
	}%
\xspace}%




















