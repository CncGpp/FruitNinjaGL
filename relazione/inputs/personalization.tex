% ============================================================================ %
%                             PACKAGE UTILIZZATI                               %
% ============================================================================ %
\usepackage{titlesec}					% Per la modifica dei titoli delle sezioni
\usepackage{tikz}
\usetikzlibrary{positioning,calc,shapes}
\usepackage{calc}						% Per effettuare calcoli
\import{inputs/}{defined-colors}
\captionsetup[table]{position=bottom}

% --------------------------------------------------------------------------- %
%	Utility                 				                                  %
% --------------------------------------------------------------------------- %
% Per ottenere i nomi di chap, section, subsection e subsubsection attuali. 
\let\Chaptermark\chaptermark
\def\chaptermark#1{\def\Chaptername{#1}\Chaptermark{#1}}
\let\Sectionmark\sectionmark
\def\sectionmark#1{\def\Sectionname{#1}\Sectionmark{#1}}
\let\Subsectionmark\subsectionmark
\def\subsectionmark#1{\def\Subsectionname{#1}\Subsectionmark{#1}}
\let\Subsubsectionmark\subsubsectionmark
\def\subsubsectionmark#1{\def\Subsubsectionname{#1}\Subsubsectionmark{#1}}


% --------------------------------------------------------------------------- %
%	Personalizzazione: Fonts				                                  %
% --------------------------------------------------------------------------- %
\iftrue
	%XeLaTeX packages per font
	\usepackage{xltxtra}
	\usepackage{fontspec} %Font package
	
	\setmainfont[Ligatures=TeX]{Minion Pro}
%	\setmainfont[Ligatures=TeX,Scale=MatchUppercase,%
%				 Path = fonts/,%
%				 BoldFont = ACaslonPro-Bold.otf,
%				 ItalicFont = ACaslonPro-Italic.otf,
%				 BoldItalicFont  = ACaslonPro-BoldItalic.otf
%				]{ACaslonPro-Regular.otf}
	\setsansfont[Scale=MatchLowercase]{Myriad Pro}
	\setmonofont[Scale=MatchLowercase]{Iosevka Term}
	
	% Per spaziare gli apostrofi.... perche Minion Pro è buggat
	\XeTeXinterchartokenstate=1
	\newXeTeXintercharclass \aposclass
	\XeTeXcharclass `' \aposclass
	\XeTeXinterchartoks \aposclass 0 = {\kern0.0pt }
	
	% Minion pro non funziona bene il textsuperscript lo devo fixare quà... è buggat su cess
	% Li ho messi in grassetto perchè mi piacevano di più.
	\setsidenotes{
		text-mark-format=\normalfont\bfseries{\textsuperscript#1}, % Stile per il marker nel testo
		note-mark-format=\textbf{#1}:,                             % Stile per il marker sulla nota a margine
		note-mark-sep=\enskip, % Set the space after the marker
	}

	\newfontfamily\HeadingsFont{Myriad Pro}
	\newfontfamily\CaptionsFont{Myriad Pro}
	\newfontfamily{\MathFont}[Scale=MatchLowercase]{Iosevka Term Semibold}
	
	% Fattore di spaziatura fra le righe
	%\setstretch{1.025}  %{Minion Pro}
	\setstretch{1.0}     %ACaslonPro
\else
\newcommand*\HeadingsFont{}
\newcommand*\CaptionsFont{}
\newcommand*\MathFont{}
\fi



% --------------------------------------------------------------------------- %
%	Personalizzazione dei titoli delle sezioni                                %
% --------------------------------------------------------------------------- %
% VARIABILI CHE REGOLANO LA SPAZIATURA
\def\spacing{1.0cm} % Spazio fra il quadrato del numero della sezione e quello della sezione
\def\indent{\oddsidemargin} % rientro delle subsection

% PERSONALIZZAZIONE DEI TITOLI DELLE SECTION
\newcommand{\titlebar}{
%	Background del testo della sezione
%	\tikz[baseline,trim left=3.1cm,trim right=3cm] {
%		\node [
%		anchor=base west,
%		%		minimum height=3.5ex,
%		minimum width=\linewidth+0.05cm,
%		fill=secTitleBackground,
%		text=secTitleBackground
%		] at (\spacing + 3.0cm-0.05cm,0) {\thesection};
%	}
	\tikz[baseline,trim left=3.1cm,trim right=3cm] {
		\node [anchor=base west,
		minimum width=3.5ex,
		fill=secNumberingBackground,
		text=secNumberingForeground] at (3cm,0) {\thesection};
	}
}
\titleformat{\section}{\color{secTitleForeground}\large\bfseries\HeadingsFont}{\titlebar}{\spacing + 0.1cm}{}


% --------------------------------------------------------------------------- %
%	Personalizzazione dei titoli delle sotto-sezioni                          %
% --------------------------------------------------------------------------- %
\setcounter{secnumdepth}{2} % Attivo la numerazione delle sub-sections
\newcommand{\subtitlebar}{
%	Background del testo della sottosezione
%	\tikz[baseline,trim left=3.1cm,trim right=3cm] {
%		\node [
%		anchor=base west,
%		minimum height=1.5ex,
%		minimum width=\linewidth-\indent+0.05cm,
%		fill=subsecTitleBackground,
%		text=subsecTitleBackground
%		] at (\spacing + \indent + 2.95cm,0) {\thesubsection};
%	}
	\tikz[baseline,trim left=3.1cm,trim right=3cm] {
		\node [anchor=base west,
		minimum height=1.5ex,
		fill=subsecNumberingBackground,
		text=subsecNumberingForeground] at (\indent + 3.0cm,0) {\thesubsection};
	}
}
\titleformat{\subsection}{\color{subsecTitleForeground}\bfseries\HeadingsFont}{\subtitlebar}{\spacing +\indent + 0.1cm}{}

% --------------------------------------------------------------------------- %
%	Personalizzazione dei titoli delle sotto-sotto-sezioni                    %
% --------------------------------------------------------------------------- %
\titleformat{\subsubsection}{\color{subsecTitleForeground}\bfseries\HeadingsFont\relsize{-0.9}}{}{}{}


% --------------------------------------------------------------------------- %
%	Aggiunta delle sotto-sotto-sotto-sezioni                                  %
% --------------------------------------------------------------------------- %
\newcommand{\subsubsubsection}[1]{\bigskip\noindent{\color{black!70!white}\small\HeadingsFont\bfseries\itshape #1} \\}


% --------------------------------------------------------------------------- %
%	Correzione spaziatura  prima e dopo le sezioni                            %
% --------------------------------------------------------------------------- %
%\titlespacing{command}{left spacing}{before spacing}{after spacing}[right]
\titlespacing\section{0pt}{14pt plus 4pt minus 2pt}{0pt plus 2pt minus 2pt}
\titlespacing\subsection{0pt}{13pt plus 4pt minus 2pt}{-2pt plus 2pt minus 2pt}
\titlespacing\subsubsection{0pt}{12pt plus 4pt minus 2pt}{-6pt plus 2pt minus 2pt}













%----------------------------------------------------------------------------------------
%	HEADERS AND FOOTERS
%----------------------------------------------------------------------------------------

\RequirePackage{scrlayer-scrpage}       % Customise head and foot regions
\setlength{\headheight}{115pt}			% Enlarge the header


% --- Creo dei nuovi comandi singoli, così è più facile metterli in questo schifo di scrheadings
% Formato pagine sinistre, in alto. Parte allineato con il testo.
\newcommand*{\headingUNO}{
	\hspace{-\marginparwidth}\hspace{-\marginparsep}%
	\normalfont 
	\begin{tikzpicture}[node distance=0cm]
	\draw node[rectangle,
	fill=secNumberingBackground,
	text=secNumberingForeground,
	draw=none,
	%	minimum height=0.55cm,
	minimum width=0.70cm
	] (A) {\small\textbf{\CaptionsFont\thechapter}};
	\draw node[
	right = of A,
	rectangle,
	fill=none,
	text=accentColor!60!black,
	draw=none,
	minimum height=0.5cm,
	minimum width=0.8cm,
	] (B) {{\scriptsize\HeadingsFont\Chaptername}};
	\end{tikzpicture}
	\hfill%
	\parbox[c][5mm][t]{0.7\linewidth}{\hfill\scriptsize\HeadingsFont\rightmark}
}
% Formato pagine destre, in alto a destra. Parte allineato con il testo.
\newcommand*{\headingDUE}{
	%\hspace{\marginparsep}\hspace{4mm}%
	\normalfont
	\makebox[\overflowingheadlen][r]{%
		\parbox[c][2mm][t]{\marginparwidth}{%\hspace{-\marginparsep}
			\scriptsize\HeadingsFont\rightmark}
		\parbox[c]{\textwidth}{\hfill 	
			\normalfont 
			\begin{tikzpicture}[node distance=0cm]
			\draw node[
			rectangle,
			fill=none,
			text=accentColor!60!black,
			draw=none,
			minimum height=0.5cm,
			minimum width=0.8cm,
			] (A) {{\scriptsize\HeadingsFont\Chaptername}};
			\draw node[right = of A,
			rectangle,
			fill=secNumberingBackground,
			text=secNumberingForeground,
			draw=none,
			%	minimum height=0.55cm,
			minimum width=0.70cm
			] (B) {\small\textbf{\CaptionsFont\thechapter}};
			\end{tikzpicture}	
		}%
	}
}


% Formato pagine sinistra, in basso a destra. Parte allineato con il testo.
\newcommand*{\headingQUATTRO}{
	\hspace{-\marginparwidth}\hspace{-\marginparsep}%
	\makebox[\overflowingheadlen][l]{%
		{\large\normalfont\textbf{\thepage}}
		\makebox[2em][c]{\rule[-1.15cm]{1pt}{1.55cm}}
		\hfill%
		{\normalfont Relazione di Grafica Interattiva -- \textbf{\color{accentColor}Giuseppe Pio Cianci}}
		\hfill
	}%
}%


% Formato pagine destre, in basso a destra. Parte allineato con il testo.
\newcommand*{\headingCINQUE}{
	%\hspace{\marginparsep}\hspace{4mm}%
	\makebox[\overflowingheadlen][r]{%
		\hfill%
		{\normalfont Relazione di Grafica Interattiva -- \textbf{\color{accentColor}Giuseppe Pio Cianci}}
		\hfill
		\makebox[2em][c]{\rule[-1.15cm]{1pt}{1.55cm}}
		{\large\normalfont\textbf{\thepage}}
		
	}%
}%

\newcommand*{\headingSETTE}{}
\newcommand*{\headingOTTO}{}

% Headings styles
\renewpagestyle{scrheadings}{
	{\headingUNO} % Pagina destra, alto a sinistra (allineato con l'inizio del testo)
	{\headingDUE} % Pagina sinistra,angolo alto a sinistra
	{\headingDUE} %tre
}{
	{\headingQUATTRO} % Pagina destra, basso a sinistra (allineato con l'inizio del testo)
	{\headingCINQUE}  % Pagina sinistra, angolo basso a sinistra
	{\headingCINQUE} %sei
}

% Pagine con stile ``plain" come quella di un nuovo capitolo
\renewpagestyle{plain.scrheadings}{
	{\headingSETTE}%
	{\headingOTTO}% Pagina sinistra - Angolo alto a sinistra
	{%nove
	}% Pagina sinistra - Angolo alto a sinistra dell'indice
}{
	{\headingQUATTRO}%
	{\headingCINQUE}% Pagina sinistra - Angolo basso a sinistra
	{%dodici
	}% Pagina sinistra - Angolo basso a sinistra dell'indice
}

\renewpagestyle{pagenum.scrheadings}{
	{tredici}%
	{quattordici}%
	{quindici}
}{
	{sedici \thepage}%
	{diciassette}%
	{diciotto}
}

% Set the default page style
\pagestyle{plain.scrheadings}





















% --------------------------------------------------------------------------- %
%	Personalizzazione: Modifico semplicemente la toc depth dei margintoc      %
% --------------------------------------------------------------------------- %
% Command to print a table of contents in the margin
\renewcommand{\margintoc}[1][\mtocshift]{ % The first parameter is the vertical offset; by default it is \mtocshift
	\begingroup
	% Set the style for section entries
	\etocsetstyle{section}
	{\parindent -5pt \parskip 0pt}
	{\leftskip 0pt}
	{\makebox[.5cm]{\etocnumber\autodot}
		\etocname\nobreak\leaders\hbox{\hbox to 1.5ex {\hss.\hss}}\hfill\nobreak
		\etocpage\par}
	{}
	% Set the style for subsection entries
	\etocsetstyle{subsection}
	{\parindent -5pt \parskip 0pt}
	{\leftskip 0pt}
	{\makebox[.5cm]{}
		\etocname\nobreak\leaders\hbox{\hbox to 1.5ex {\hss.\hss}}\hfill\nobreak
		\etocpage\par}
	{}
	% Set the global style of the toc
	%\etocsettocstyle{}{}
	%\etocsettocstyle{\normalfont\sffamily\normalsize}{}
	%\etocsettocstyle{\usekomafont{section}\small}{}
	\etocsettocstyle{\usekomafont{chapterentry}\small}{}
	\etocsetnexttocdepth{section}
	% Print the table of contents in the margin
	\marginnote[#1]{\localtableofcontents}
	\endgroup
}



