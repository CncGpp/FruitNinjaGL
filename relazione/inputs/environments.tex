% ============================================================================ %
%                      DEFINIZIONE DEI NUOVI AMBIENTI                          %
% ============================================================================ %
% --------------------------------------------------------------------------- %
%	Creazione dei BOX                                                         %
% --------------------------------------------------------------------------- %
% usa l'opzione [frametitle=To Do] per dargi un titolo

% __ ESEMPIO __
\theoremstyle{kaoexample}
\declaretheorem[
	style=kaoexample,
	name=Esempio,
	refname={esempio,esempi},
	Refname={Esempio,Esempi},
	numberwithin=section,
	mdframed={
		style=mdfkao,
		backgroundcolor=exampleColor!20!white,
		frametitlebackgroundcolor=exampleColor!20!white,
	},
]{exampleBox}

% __ OSSERVA __
\theoremstyle{kaoexample}
\declaretheorem[
style=kaoexample,
name=Osserva,
refname={osservazione,osservazioni},
Refname={Osservazione,Osservazioni},
numberwithin=section,
mdframed={
	style=mdfkao,
	backgroundcolor=observeColor!20!white,
	frametitlebackgroundcolor=observeColor!20!white,
},
]{observeBox}


% __ IDEA __
\theoremstyle{kaoexample}
\declaretheorem[
style=kaoexample,
name=Idea,
refname={idea,idee},
Refname={Idea,Idee},
numberwithin=section,
mdframed={
	style=mdfkao,
	backgroundcolor=ideaColor!20!white,
	frametitlebackgroundcolor=ideaColor!20!white,
},
]{ideaBox}

% __ REMEMBER __
\theoremstyle{kaoexample}
\declaretheorem[
style=kaoexample,
name=Ricorda,
refname={ricorda,ricorda},
Refname={Ricorda,Ricorda},
numberwithin=section,
mdframed={
	style=mdfkao,
	backgroundcolor=rememberColor!20!white,
	frametitlebackgroundcolor=rememberColor!20!white,
},
]{rememberBox}

% __ WARNING __
\theoremstyle{kaoexample}
\declaretheorem[
style=kaoexample,
name=Attenzione,
refname={attenzione,attenzione},
Refname={Attenzione,Attenzione},
numberwithin=section,
mdframed={
	style=mdfkao,
	backgroundcolor=warningColor!20!white,
	frametitlebackgroundcolor=warningColor!20!white,
},
]{warningBox}

% __ IMPORTANT __
\theoremstyle{kaoexample}
\declaretheorem[
style=kaoexample,
name=Importante,
refname={importante,importante},
Refname={Importante,Importante},
numberwithin=section,
mdframed={
	style=mdfkao,
	backgroundcolor=importantColor!45!white,
	frametitlebackgroundcolor=importantColor!45!white,
},
]{importantBox}

% __ EMPTYBOX __
\newmdenv[
	style=kaoboxstyle,
	skipabove=\topskip,
	skipbelow=.5\topskip,
	backgroundcolor=black!5!white,
	frametitlebackgroundcolor=black!5!white,
]{emptyBox}























			        %Definizione di nuovi box


% --------------------------------------------------------------------------- %
%	Environments: Ambienti compatti				                              %
% --------------------------------------------------------------------------- %
% Liste compatte ma con spazi prima e dopo
\newenvironment{itemize*}%
{\begin{itemize}[itemsep=0pt,parsep=0pt]}%
	{\end{itemize}}
\newenvironment{enumerate*}%
{\begin{enumerate}[itemsep=0pt,parsep=0pt]}%
	{\end{enumerate}}

% Liste compatte senza spazi prima e dopo
\newenvironment{compactitemize} {\begin{itemize}[noitemsep,topsep=-5pt,parsep=0pt,partopsep=-6pt]} {\end{itemize}}
\newenvironment{compactenumerate} {\begin{enumerate}[noitemsep,topsep=-5pt,parsep=0pt,partopsep=-6pt]} {\end{enumerate}}

% Center compatto
\newenvironment{compactcenter}[1][4pt]{%
	\setlength\topsep{#1}%
	\setlength\parskip{#1}%
	\par\centering}{\par\noindent\ignorespacesafterend}

% Equazioni compatte con numerazione
\newenvironment{compactequation}[1][4pt]{
	\begingroup%
	\setlength{\abovedisplayskip}{#1}%
	\setlength{\belowdisplayskip}{#1}%
	\begin{equation}}%
{\end{equation}\endgroup\noindent\ignorespacesafterend}%
% Equazioni compatte senza numerazione
\newenvironment{compactequation*}[1][4pt]{
	\begingroup%
	\setlength{\abovedisplayskip}{#1}%
	\setlength{\belowdisplayskip}{#1}%
	\begin{equation*}}%
{\end{equation*}\endgroup\noindent\ignorespacesafterend}%


% --------------------------------------------------------------------------- %
%	Environments: ambienti extra				                              %
% --------------------------------------------------------------------------- %
\newenvironment{chap-intro}{\begin{center}\begin{minipage}{.85\linewidth}} {\end{minipage}\end{center}}

% --------------------------------------------------------------------------- %
%	Environments: testo evidenziato                                           %
% --------------------------------------------------------------------------- %
% Da https://tex.stackexchange.com/questions/24036/fixing-a-code-to-highlight-formulas-and-text-on-several-lines
% usage \highlight{<colore_bg> default yellow, <<opzioni tikz aggiuntive e opzionali>>}
\usepackage{soul}
\usepackage{tikz}
\usetikzlibrary{calc}
\usepackage{amsmath}
\usepackage{xcolor}

\makeatletter
\newcommand{\defhighlighter}[3][]{%
	\tikzset{every highlighter/.style={color=#2, fill opacity=#3, #1}}%
}

\defhighlighter{yellow}{.5}

\newcommand{\highlight@DoHighlight}{
	\fill[
	outer sep = -15pt, inner sep = 0pt, 
	every highlighter, this highlighter 
	](
	$(begin highlight)+(0,8pt)$) rectangle ($(end highlight)+(0,-3pt)$
	);
}

\newcommand{\highlight@BeginHighlight}{
	\coordinate (begin highlight) at (0,0) ;
}

\newcommand{\highlight@EndHighlight}{
	\coordinate (end highlight) at (0,0) ;
}

\newdimen\highlight@previous
\newdimen\highlight@current

\DeclareRobustCommand*\highlight[1][]{%
	\tikzset{this highlighter/.style={#1}}%
	\SOUL@setup
	%
	\def\SOUL@preamble{%
		\begin{tikzpicture}[overlay, remember picture]
		\highlight@BeginHighlight
		\highlight@EndHighlight
		\end{tikzpicture}%
	}%
	%
	\def\SOUL@postamble{%
		\begin{tikzpicture}[overlay, remember picture]
		\highlight@EndHighlight
		\highlight@DoHighlight
		\end{tikzpicture}%
	}%
	%
	\def\SOUL@everyhyphen{%
		\discretionary{%
			\SOUL@setkern\SOUL@hyphkern
			\SOUL@sethyphenchar
			\tikz[overlay, remember picture] \highlight@EndHighlight ;%
		}{}{%
			\SOUL@setkern\SOUL@charkern
		}%
	}%
	%
	\def\SOUL@everyexhyphen##1{%
		\SOUL@setkern\SOUL@hyphkern
		\hbox{##1}%
		\discretionary{%
			\tikz[overlay, remember picture] \highlight@EndHighlight ;%
		}{}{%
			\SOUL@setkern\SOUL@charkern
		}%
	}%
	%
	\def\SOUL@everysyllable{%
		\begin{tikzpicture}[overlay, remember picture]
		\path let \p0 = (begin highlight), \p1 = (0,0) in \pgfextra
		\global\highlight@previous=\y0
		\global\highlight@current =\y1
		\endpgfextra (0,0) ;
		\ifdim\highlight@current < \highlight@previous
		\highlight@DoHighlight
		\highlight@BeginHighlight
		\fi
		\end{tikzpicture}%
		\the\SOUL@syllable
		\tikz[overlay, remember picture] \highlight@EndHighlight ;%
	}%
	\SOUL@
}
\makeatother



















